\section{Media Device}\label{sec:Device Types / Media Device}

The virtio media device follow the same model (and structures) as V4L2. It
can be used to virtualize cameras, codec devices, or any other device
supported by V4L2. The complete definition of V4L2 structures and ioctls can
be found under the
\href{https://www.kernel.org/doc/html/latest/userspace-api/media/index.html}{V4L2 UAPI documentation}.

V4L2 is a UAPI that allows a less privileged entity (user-space) to use video
hardware exposed by a more privileged entity (the kernel). Virtio-media is an
encapsulation of this API into virtio, turning it into a virtualization API
for all classes of video devices supported by V4L2, where the host plays the
role of the kernel and the guest the role of user-space.

The host is therefore responsible for presenting a virtual device that behaves
like an actual V4L2 device, which the guest can control.

Note that virtio-media does not require the use of a V4L2 device driver or of
Linux on the host or guest side - V4L2 is only used as a host-guest protocol,
and both sides are free to convert it from/to any model that they wish to use.

This section relies on definitions from
\href{https://www.kernel.org/doc/html/latest/userspace-api/media/index.html}{V4L2 UAPI documentation}.

\subsection{Device ID}\label{sec:Device Types / Media Device / Device ID}

42

\subsection{Virtqueues}\label{sec:Device Types / Media Device / Virtqueues}

\begin{description}
\item[0] commandq - used for driver commands and device responses to these
commands.
\item[1] eventq - used for events sent by the device to the driver.
\end{description}

\devicenormative{\subsubsection}{Virtqueues}{Device Types / Media Device / Virtqueues}

The device MUST return the descriptor chains it receives as soon as
possible, and must never hold to them for indefinite periods of time.

\drivernormative{\subsubsection}{Virtqueues}{Device Types / Media Device / Virtqueues}

The driver MUST re-queue the descriptor chains returned by the device as soon
as possible, and must never hold on them for indefinite periods of time.

\subsection{Feature Bits}\label{sec:Device Types / Media Device / Feature Bits}

None

\subsection{Device Configuration Layout}\label{sec:Device Types / Media Device / Device Configuration Layout}

The video device configuration space uses the following layout:

\begin{lstlisting}
struct virtio_media_config {
    le32 device_caps;
    le32 device_type;
    u8 card[32];
};
\end{lstlisting}

\begin{description}
\item[\field{device_caps}] (driver-read-only) flags representing the device
capabilities as used in
\href{https://www.kernel.org/doc/html/v4.9/media/uapi/v4l/vidioc-querycap.html#c.v4l2_capability}{struct v4l2_capabilities}.
Corresponds with the \field{device_caps} field in the \textit{struct video_device}.
\item[\field{device_type}] (driver-read-only) informs the driver of the type
of the video device. Corresponds with the \field{vfl_devnode_type} field of the device.
\item[\field{card}] (driver-read-only) name of the device, a unique NUL-terminated
UTF-8 string. Corresponds with the \field{card} field of the \textit{struct v4l2_capability}.
\end{description}

\subsection{Device Initialization}

\begin{enumerate}
\item The driver reads the \field{device_caps} and \field{device_type} fields
from the configuration layout to identify the device.
\item The driver sets up the \field{commandq} and \field{eventq}.
\item The driver may open a session to use the device and send V4L2 ioctls in
order to receive more information about the device, such as formats or controls.
\end{enumerate}

\subsection{Device Operation}\label{sec:Device Types / Media Device / Device Operation}

Commands are queued on the command queue by the driver for the device to
process. The errors returned by each command are standard
\href{https://www.kernel.org/doc/html/latest/userspace-api/media/gen-errors.html}{Linux kernel error codes}.
For instance, a command that contains invalid options will return \textit{EINVAL}.

Events are sent on the event queue by the device for the driver to handle.

\subsubsection{Command Virtqueue}

\paragraph{Device Operation: Command headers}

\begin{lstlisting}
#define VIRTIO_MEDIA_CMD_OPEN 1
#define VIRTIO_MEDIA_CMD_CLOSE 2
#define VIRTIO_MEDIA_CMD_IOCTL 3
#define VIRTIO_MEDIA_CMD_MMAP 4
#define VIRTIO_MEDIA_CMD_MUNMAP 5

/* Header for all virtio commands from the driver to the device on the commandq. */
struct virtio_media_cmd_header {
	u32 cmd;
	u32 __padding;
};

/* Header for all virtio responses from the device to the driver on the commandq. */
struct virtio_media_resp_header {
	u32 status;
	u32 __padding;
};
\end{lstlisting}

A request part has, or consists of, a command header \textit{virtio_media_cmd_header}
containing the following device-readable field:

\begin{description}
\item[\field{cmd}] specifies a device request type (VIRTIO_MEDIA_CMD_*).
\end{description}

A response part has, or consists of, a response header \textit{virtio_media_resp_header}
containing the following device-writable field:

\begin{description}
\item[\field{status}] indicates a device request status.
\end{description}

The status field can take 0 if the command was successful, or one of the
standard Linux error codes if it was not.

\drivernormative{\paragraph}{Device Operation: Command Virtqueue: Sessions}{Device Types / Media Device / Device Operation / Command Virtqueue}

Sessions are how the device is multiplexed, allowing several distinct works to
take place simultaneously. The driver needs to open a session before it can
perform any useful operation on the device.

\paragraph{Device Operation: Open device}

\textbf{VIRTIO_MEDIA_CMD_OPEN} Command for creating a new session.

This is the equivalent of calling \textit{open} on a V4L2 device node.
The driver users \textit{virtio_media_cmd_open} to send an open request.

\begin{lstlisting}
struct virtio_media_cmd_open {
    struct virtio_media_cmd_header hdr;
};
\end{lstlisting}

The device responds to \textit{VIRTIO_MEDIA_CMD_OPEN} with \textit{virtio_media_resp_open}.

\begin{lstlisting}
struct virtio_media_resp_open {
    struct virtio_media_resp_header hdr;
    u32 session_id;
    u32 __padding;
};
\end{lstlisting}

\begin{description}
\item[\field{session_id}] specifies an identifier for the current session. The
identifier can be used to perform other commands on the session, notably ioctls.
\end{description}

\drivernormative{\subparagraph}{Device Operation: Open device}{Device Types / Media Device / Device Operation / Open device}

\begin{itemize}
\item Upon success, the driver MUST set a \field{session_id} in \textit{virtio_media_resp_open}
to an integer that is NOT used by any other open session.
\end{itemize}

\paragraph{Device Operation: Close device}

\textbf{VIRTIO_MEDIA_CMD_CLOSE} Command for closing an active session.

This is the equivalent of calling \textit{close} on a previously opened V4L2
device node. All resources associated with this session will be freed.

This command does not require a response from the device.

\begin{lstlisting}
struct virtio_media_cmd_close {
    struct virtio_media_cmd_header hdr;
    u32 session_id;
    u32 __padding;
};
\end{lstlisting}

\begin{description}
\item[\field{session_id}] specifies an identifier for the session to close.
\end{description}

\drivernormative{\subparagraph}{Device Operation: Close device}{Device Types / Media Device / Device Operation / Close device}

The session ID SHALL NOT be used again after queueing this command.

\paragraph{Device Operation: V4L2 ioctls}

\textbf{VIRTIO_MEDIA_CMD_IOCTL} Command for executing an ioctl on an open
session.

This command asks the device to run one of the `VIDIOC_*` ioctls on the active
session.

\begin{lstlisting}
struct virtio_media_cmd_ioctl {
    struct virtio_media_cmd_header hdr;
    u32 session_id;
    u32 code;
    /* Followed by the relevant ioctl payload as defined in the macro */
};
\end{lstlisting}

\begin{description}
\item[\field{session_id}] specifies an identifier of thesession to run the ioctl on.
\item[\field{code}] specifies the code of the \field{VIDIOC_*} ioctl to run.
\end{description}

The code is extracted from the
\href{https://www.kernel.org/doc/html/latest/userspace-api/media/v4l/videodev.html}{videodev2.h},
header file. The file defines the ioctl's codes, type of payload, and
direction. The code consists of the second argument of the \field{_IO*} macro.

For example, the \textit{VIDIOC_G_FMT} is defined as follows:

\begin{lstlisting}
#define VIDIOC_G_FMT _IOWR('V',  4, struct v4l2_format)
\end{lstlisting}

This means that its ioctl code is \textit{4}, that its payload is a
\textit{struct v4l2_format}, and that its direction is \textit{WR} (i.e., the
payload is written by both the driver and the device).
See Section \ref{sec:Device Types / Media Device / V4L2 ioctls / Ioctls payload}
for more information about the direction of ioctls.

The payload layout is always a 64-bit representation of the corresponding
V4L2 structure, irrespective of the host and guest architecture.

The device responds to \textit{VIRTIO_MEDIA_CMD_IOCTL} with \textit{virtio_media_resp_ioctl}.

\begin{lstlisting}
struct virtio_media_resp_ioctl {
    struct virtio_media_resp_header hdr;
    /* Followed by the ioctl payload as defined in the macro */
};
\end{lstlisting}

\subparagraph{Ioctls payload}\label{sec:Device Types / Media Device / V4L2 ioctls / Ioctls payload}

Each ioctl has a payload, which is defined by the third argument of the
\field{_IO*} macro defining it. 

The payload of an ioctl in the descriptor chain follows the command structure,
the reponse structure, or both depending on the direction:

\begin{itemize}
\item \textbf{_IOR} is read-only for the driver, meaning the payload
follows the response in the device-writable section of the descriptor chain.
\item \textbf{_IOW} is read-only for the device, meaning the payload
follows the command in the driver-writable section of the descriptor chain.
\item \textbf{_IOWR} is writable by both the device and driver,
meaning the payload must follow both the command in the driver-writable section
of the descriptor chain, and the response in the device-writable section.
\end{itemize}

A common optimization for \textit{WR} ioctls is to provide the payload using
descriptors that both point to the same buffer. This mimics the behavior of
V4L2 ioctls where the data is only passed once and used as both input and
output by the kernel.

\devicenormative{\subparagraph}{Device Operation: V4L2 ioctls}{Device Types / Media Device / Device Operation / V4L2 ioctls}

In case of success of a device-writable ioctl, the device MUST always write the
payload in the device-writable part of the descriptor chain.

In case of failure of a device-writable ioctl, the device is free to write the
payload in the device-writable part of the descriptor chain or not. Some errors
may still result in the payload being updated, and in this case the device is
expected to write the updated payload. If the device has not written the
payload after an error, the driver MUST assume that the payload has not been
modified.

\subparagraph{Handling of pointers in ioctl payload}

A few structures used as ioctl payloads contain pointers to further
data needed for the ioctl. There are notably:

\begin{itemize}
\item The \field{planes} pointer of
\href{https://www.kernel.org/doc/html/latest/userspace-api/media/v4l/buffer.html#struct-v4l2-buffer}{struct v4l2_buffer},
which size is determined by the length member.
\item The \field{controls} pointer of \textit{struct v4l2_ext_controls}, which
size is determined by the count member.
\end{itemize}

If the size of the pointed area is determined to be non-zero, then the main
payload is immediately followed by the pointed data in their order of
appearance in the structure, and the pointer value itself is ignored by the
device, which must also return the value initially passed by the driver.

\subparagraph{Handling of pointers to userspace memory}

A few pointers are special in that they point to userspace memory. They are:

\begin{itemize}
\item The \field{m.userptr} member of \textit{struct v4l2_buffer} and
\href{https://www.kernel.org/doc/html/latest/userspace-api/media/v4l/buffer.html#struct-v4l2-plane}{struct v4l2_plane}
(technically an unsigned long, but designated a userspace address).
\item The \field{ptr} member of \textit{struct v4l2_ext_ctrl}.
\end{itemize}

These pointers can cover large areas of scattered memory, which has the
potential to require more descriptors than the virtio queue can provide. For
these particular pointers only, a list of \textit{struct virtio_media_sg_entry}
that covers the needed amount of memory for the pointer is used instead of
using descriptors to map the pointed memory directly.

For each such pointer to read, the device reads as many SG entries as needed
to cover the length of the pointed buffer, as described by its parent
structure (length member of \textit{struct v4l2_buffer} or
\textit{struct v4l2_plane} for buffer memory, and size member of
\textit{struct v4l2_ext_control} for control data).

Since the device never needs to modify the list of SG entries, it is only
provided by the driver in the device-readable section of the descriptor chain,
and not repeated in the device-writable section, even for WR ioctls.

\subparagraph{Unsupported ioctls}

A few ioctls are replaced by other, more suitable mechanisms.

\begin{itemize}
\item \textit{VIDIOC_QUERYCAP} is replaced by reading the configuration area
(see \ref{sec:Device Types / Media Device / Device Configuration Layout}).
\item \textit{VIDIOC_DQBUF} is replaced by a dedicated event
(see \ref{sec:Device Types / Media Device / Device Operation / Dequeue buffer}).
\item \textit{VIDIOC_DQEVENT} is replaced by a dedicated event
(see \ref{sec:Device Types / Media Device / Device Operation / Emit an event}).
\item \textit{VIDIOC_G_JPEGCOMP} and \textit{VIDIOC_S_JPEGCOMP} are deprecated
and replaced by the controls of the JPEG class.
\item \textit{VIDIOC_LOG_STATUS} is a guest-only operation and shall not be
implemented by the host.
\end{itemize}

\devicenormative{\subparagraph}{Device Operation: Unsupported ioctls}{Device Types / Media Device / Device Operation / Unsupported ioctls}

If being requested an unsupported ioctl, the device MUST return the same
response as it would for an unknown ioctl, i.e. \textit{ENOTTY}.

\paragraph{Device Operation: Mapping a MMAP buffer}

\textbf{VIRTIO_MEDIA_CMD_MMAP} Command for mapping a MMAP buffer into the
guest's address space.

\begin{lstlisting}
#define VIRTIO_MEDIA_MMAP_FLAG_RW (1 << 0)

struct virtio_media_cmd_mmap {
	struct virtio_media_cmd_header hdr;
	u32 session_id;
	u32 flags;
	u64 offset;
};
\end{lstlisting}

\begin{description}
\item[\field{flags}] is the set of flags for the mapping. \field{VIRTIO_MEDIA_MMAP_FLAG_RW}
can be set if a read-write mapping is desired. Without this flag the mapping
will be read-only.
\item[\field{offset}] corresponds to the \field{mem_offset} field of the
\textit{union virtio_v4l2_plane_mem_backing} for the plane to map. This field
cane be obtained using the \textit{VIDIOC_QUERYBUF} ioctl.
\end{description}

The device responds to \textit{VIRTIO_MEDIA_CMD_MMAP} with \textit{virtio_media_resp_mmap}.

\begin{lstlisting}
struct virtio_media_resp_mmap {
    struct virtio_media_resp_header hdr;
    u64 addr;
    u64 len;
};
\end{lstlisting}

\begin{description}
\item[\field{addr}] device physical address of the start of the mapping.
\item[\field{len}] length of the mapping as indicated by the \textit{struct v4l2_plane}
the buffer belongs to.
\end{description}

\paragraph{Device Operation: Unmapping a MMAP buffer}

\textbf{VIRTIO_MEDIA_CMD_MUNMAP} Unmap a MMAP buffer previously mapped using \field{VIRTIO_MEDIA_CMD_MMAP}.

\begin{lstlisting}
struct virtio_media_cmd_munmap {
    struct virtio_media_cmd_header hdr;
    u64 guest_addr;
};
\end{lstlisting}

\begin{description}
\item[\field{guest_addr}] guest physical address at which the buffer has been previously mapped.
\end{description}

The device responds to \textit{VIRTIO_MEDIA_CMD_MUNMAP} with \textit{virtio_media_resp_munmap}.

\begin{lstlisting}
struct virtio_media_resp_munmap {
    struct virtio_media_resp_header hdr;
};
\end{lstlisting}

\devicenormative{\subparagraph}{Device Operation: Unmapping a MMAP buffer}{Device Types / Media Device / Device Operation / Unmapping a MMAP buffer}

The device MUST keep mappings performed using \textit{VIRTIO_MEDIA_CMD_MMAP}
valid until \textit{VIRTIO_MEDIA_CMD_MUNMAP} is called, even if the buffers or
session they belong to are released or closed by the guest.

\paragraph{Device Operation: Memory Types}

The semantics of the three V4L2 memory types (\textit{MMAP}, \textit{USERPTR}
and \textit{DMABUF}) can easily be mapped to a guest/host context.

\subparagraph{MMAP}

In virtio-media, \textit{MMAP} buffers are provisioned by the host, just like
they are by the kernel in regular V4L2. Similarly to how userspace can map a
\textit{MMAP} buffer into its address space using mmap and munmap, the
virtio-media driver can map host buffers into the guest space by queueing the
\textit{struct virtio_media_cmd_mmap} and \textit{struct virtio_media_cmd_munmap}
commands to the commandq.

\subparagraph{USERPTR}

In virtio-media, \textit{USERPTR} buffers and provisioned by the guest, just
like they are by userspace in regular V4L2. Instances of \textit{struct v4l2_buffer}
and \textit{struct v4l2_plane} of this type are followed by a series of
descriptors mapping the buffer backing memory in guest space.

For the host convenience, the backing memory must start with a new descriptor
- this allows the host to easily map the buffer memory to render into it
instead of having to do a copy.

The host must not alter the pointer values provided by the guest, i.e.
\field{the m.userptr} member of \textit{struct v4l2_buffer} and
\textit{struct v4l2_plane} must be returned to the guest with the same value
as it was provided.

\subparagraph{DMABUF}

In virtio-media, \textit{DMABUF} buffers are provisioned by a virtio object,
just like they are by a \textit{DMABUF} in regular V4L2. Virtio objects are
16-bytes UUIDs and do not fit in the placeholders for file descriptors, so
they follow their embedding data structure as needed and the device must
leave the V4L2 structure placeholder unchanged.

Contrary to \textit{USERPTR} buffers, virtio objects UUIDs need to be added in
both the device-readable and device-writable section of the descriptor chain.

Host-allocated buffers with the \textit{V4L2_MEMORY_MMAP} memory type can also
be exported as virtio objects for use with another virtio device using the
\textit{VIDIOC_EXPBUF} ioctl. The fd placefolder of \textit{v4l2_exportbuffer}
means that space for the UUID needs to be reserved right after that structure

\subsubsection{Event Virtqueue}

Events are a way for the device to inform the driver about asynchronous events
that it should know about. In virtio-media, they are used as a replacement for
the \textit{VIDIOC_DQBUF} and \textit{VIDIOC_DQEVENT} ioctls and the polling
mechanism, which would be impractical to implement on top of virtio.

\paragraph{Device Operation: Event header}

\begin{lstlisting}
#define VIRTIO_MEDIA_EVT_ERROR 0
#define VIRTIO_MEDIA_EVT_DQBUF 1
#define VIRTIO_MEDIA_EVT_EVENT 2

/* Header for events queued by the device for the driver on the eventq. */
struct virtio_media_event_header {
    u32 event;
    u32 session_id;
};
\end{lstlisting}

\begin{description}
\item[\field{event}] one of \field{VIRTIO_MEDIA_EVT_*}.
\item[\field{session_id}] ID of the session the event applies to.
\end{description}

\paragraph{Device Operation: Device-side error}

\textbf{VIRTIO_MEDIA_EVT_ERROR} Upon receiving this event, the session
mentioned in the header is considered corrupted and closed.

\begin{lstlisting}
struct virtio_media_event_error {
    struct virtio_media_event_header hdr;
    u32 errno;
    u32 __padding;
};
\end{lstlisting}

\begin{description}
\item[\field{errno}] error code describing the kind of error that occurred.
\end{description}

\paragraph{Device Operation: Dequeue buffer}
\label{sec:Device Types / Media Device / Device Operation / Dequeue buffer}

\textbf{VIRTIO_MEDIA_EVT_DQBUF} Signals that a buffer is not being used anymore
by the device and is returned to the driver.

A \textit{struct virtio_media_event_dqbuf} event is queued on the eventq by the
device every time a buffer previously queued using the \textit{VIDIOC_QBUF}
ioctl is done being processed and can be used by the driver again. This is like
an implicit \textit{VIDIOC_DQBUF} ioctl.

\begin{lstlisting}
struct virtio_media_event_dqbuf {
    struct virtio_media_event_header hdr;
    struct v4l2_buffer buffer;
    struct v4l2_plane planes[VIDEO_MAX_PLANES];
};
\end{lstlisting}

\begin{description}
\item[\field{buffer}] \textit{struct v4l2_buffer} describing the buffer that has been dequeued.
\item[\field{planes}] array of \textit{struct v4l2_plane} containing the plane infromation for multi-planar buffers.
\end{description}

Pointer values in the \textit{struct v4l2_buffer} and \textit{struct v4l2_plane}
are meaningless and must be ignored by the driver. It is recommended that the
device sets them to NULL in order to avoid leaking potential host addresses.

Note that in the case of a \field{USERPTR} buffer, the \textit{struct v4l2_buffer}
used as event payload is not followed by the buffer memory: since that memory
is the same that the driver submitted with the \textit{VIDIOC_QBUF}, it would
be redundant to have it here.

\paragraph{Device Operation: Emit an event}
\label{sec:Device Types / Media Device / Device Operation / Emit an event}

\textbf{VIRTIO_MEDIA_EVT_EVENT} Signals that a V4L2 event has been emitted for a session.

A \textit{struct virtio_media_event_event} event is queued on the eventq by the
device every time an event the driver previously subscribed to using the
\textit{VIDIOC_SUBSCRIBE_EVENT} ioctl has been signaled. This is like an
implicit \textit{VIDIOC_DQEVENT} ioctl.

\begin{lstlisting}
struct virtio_media_event_event {
    struct virtio_media_event_header hdr;
    struct v4l2_event event;
};
\end{lstlisting}

\begin{description}
\item[\field{event}] \textit{struct v4l2_event} describing the event that occurred.
\end{description}
